\documentclass[12pt, a4paper]{article}
\usepackage[utf8]{inputenc}
\usepackage[T1]{fontenc}
\usepackage{amsmath}
\usepackage{amssymb}
\usepackage{amsthm}
\usepackage{geometry}
\usepackage{color}
\usepackage{listings}
\usepackage{hyperref}
\usepackage[dvipsnames]{xcolor}
\usepackage[polish]{babel}
\usepackage{graphicx}
\usepackage{subcaption}
\usepackage{float}

\geometry{
 a4paper,
 margin=1in,
}

\title{Sprawozdanie z Implementacji Algorytmu Triangulacji Wielokątów $y$-Monotonicznych}
\author{Artur Radwański - grupa 4}
\date{24 listopada 2025}

\begin{document}

\maketitle

\section{Wstęp}
\subsection{Cel ćwiczenia}
Celem ćwiczenia była implementacja algorytmu wykrywania przecięć odcinków na płaszczyźnie kartezjańskiej.

\subsection{Implementacja}
Jako struktury zdarzeń oraz struktury stanu, użyto SortedSet.

W strukturze zdarzeń trzymano krotki w postaci ((x,y), typ, id1, id2)
gdzie: 
\begin{itemize}
    \item (x,y) - współrzędne punktu
    \item typ - string wskazujący typ zdarzenia
    \item id1, id2 - indeks odcinka w oryginalnej tablicy, id2 jest opcjonalne i jest dodawane tylko jeśli zdarzenie to przecięcie
\end{itemize}

W strukturze stanu trzymano krotki (Odcinek, id) gdzie



\end{document}