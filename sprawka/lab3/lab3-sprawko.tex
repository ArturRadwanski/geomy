\documentclass[12pt, a4paper]{article}
\usepackage[utf8]{inputenc}
\usepackage[T1]{fontenc}
\usepackage{amsmath}
\usepackage{amssymb}
\usepackage{amsthm}
\usepackage{geometry}
\usepackage{color}
\usepackage{listings}
\usepackage{hyperref}
\usepackage[dvipsnames]{xcolor}
\usepackage[polish]{babel}
\usepackage{graphicx}
\usepackage{subcaption}
\usepackage{float}

\geometry{
 a4paper,
 margin=1in,
}

\title{Sprawozdanie z Implementacji Algorytmu Triangulacji Wielokątów $y$-Monotonicznych}
\author{Artur Radwański - grupa 4}
\date{24 listopada 2025}

\begin{document}

\maketitle

\section{Realizacja Ćwiczenia}
\subsection{Cel}

Celem ćwiczenia była implementacja kluczowych etapów przetwarzania wielokątów w geometrii obliczeniowej: sprawdzenie
$y$-monotoniczności, kategoryzacja wierzchołków oraz właściwa triangulacja wielokąta $y$-monotonicznego przy użyciu
algorytmu opisanego na wykładzie.

\subsection{Wstęp teoretyczny}
Wielokąt jest $\mathbf{y}$-monotoniczny, jeśli każda prosta pozioma przecina go w co najwyżej dwóch punktach.
Taki wielokąt można podzielić na dwa łańcuchy z których każdy jest przecinany co najwyżej raz przez każdą poziomą prostą.
\medskip

\noindent
Wierzchołki dowolnego wielokąta możemy skategoryzować ze względu na położenie sąsiadów w następujący sposób:
\begin{itemize}
    \item  \textbf{\color{OliveGreen}Początkowy} - Sąsiedzi poniżej, kąt wypukły ($< 180^\circ$).
    \item \textbf{\color{red}Końcowy} - Sąsiedzi powyżej, kąt wypukły ($< 180^\circ$).
    \item \textbf{\color{blue}Łączący} - Sąsiedzi powyżej, kąt wklęsły ($> 180^\circ$).
    \item \textbf{\color{cyan}Dzielący} - Sąsiedzi poniżej, kąt wklęsły ($> 180^\circ$).
    \item \textbf{\color{pink}Prawidłowy} - Jeden sąsiad powyżej, drugi poniżej.
\end{itemize}
Wielokąt jest $y$-monotoniczny wtedy i tylko wtedy gdy nie posiada żadnego wierzchołka łączącego, ani dzielącego.
\medskip

\noindent
Algorytm triangulacji opisany jest następująco:
\begin{itemize}
    \item Przydzielamy każdy z wierzchołków do łańcucha
    \item Sortujemy wierzchołki względem współrzędnej $y$
    \item Umieszczamy dwa pierwsze wierzchołki na stosie
    \item Jeśli sprawdzany wierzchołek należy do innego łańucha od tego ze szczytu stosu, to łączymy go ze wszystkimi wierzchołkami
    na stosie, następnie wstawiamy na stos wierzchołek który był na szczycie oraz ten który był właśnie przetwarzany
    \item Jeśli sprawdzany wierzchołek należy do tego samego łańcucha, to analizujemy trójkąt jaki tworzy z wierzchołkami na
    szczycie stosu, jeśli zawiera się w tym wielokącie, to zdejmujemy jeden wierzchołek ze stosu i powtarzamy ten krok,
    w przeciwnym wypadku umieszczamy badane wierzchołki na stosie i przechodzimy do następnego wierzchołka
\end{itemize}

\subsection{Opis implementacji}

\paragraph{Sprawdzanie $y$-monotoniczności:}
Funkcja zakłada, że współrzędne wierzchołków są podane w postaci tablicy krotek i punkty są posortowane w kolejności przeciwnej
do ruchu wskazówek zegara. Funkcja najpierw znajduje wierzchołek o najwyższym i najniższym $y$, następnie dzieli wielokąt na 
dwa łańcuchy zaczynające i kończące się w tych wierzchołkach, następnie sprawdza czy oba są monotoniczne, jeśli tak, to
wielokąt jest $y$-monotoniczny.

\paragraph{Klasyfikacja wierzchołków:}
Funkcja iteruje przez wszystkie wierzchołki sprawdzając położenie sąsiadów oraz wypukłość kąta, przy sprawdzanym wierzchołku.
Do sprawdzania wypukłości kąta, jest używana metoda wyznacznika macierzy 3x3.

\paragraph{Triangulacja:}
Funkcja realizuje algorytm opisany we wstępie teoretycznym. Po podzieleniu wielokąta na dwa łańcuchy, wierzchołki są sortowane
przez scalanie. W ramach stosu użyto struktury deque z biblioteki queque. Do sprawdzania czy dany wierzchołek należy do wielokąta
użyto metody obliczania wyznacznika macierzy 3x3. Przeanalizowano skuteczność algorytmu dla różnych wielokątów.


\section{Wyniki}
\subsection{Klasyfikacja wierzchołków}
Poniżej dwie grafiki pokazujące na przykładzie jak wygląda klasyfikacja wierzchołków w wielokącie:

Legenda:
\begin{itemize}
    \item  \textbf{\color{OliveGreen}Początkowy}
    \item \textbf{\color{red}Końcowy}
    \item \textbf{\color{blue}Łączący} 
    \item \textbf{\color{cyan}Dzielący}
    \item \textbf{\color{pink}Prawidłowy}
\end{itemize}

\begin{figure}[H]
  \centering
  \begin{subfigure}[b]{0.4\linewidth}
    \includegraphics[width=\linewidth]{images/nony_colors.png}
    \caption{Wielokąt dowolny}
  \end{subfigure}
  \begin{subfigure}[b]{0.4\linewidth}
    \includegraphics[width=\linewidth]{images/y-colors.png}
    \caption{Wielokąt $y$-monotoniczny}
  \end{subfigure}
  \caption{Klasyfikacja wierzchołków}
  \label{fig:coffee}
\end{figure}

\subsection{triangulacja}
Algorytm triangulacji wielokątów $y$-monotonicznych został przetestowany na zróżnicowanym zestawie figur, aby zweryfikować jego poprawność. Testy wizualizują przekątne (zaznaczone kolorem czerwonym) dodane przez algorytm.

\paragraph{Wyniki Triangulacji dla Różnych Wielokątów Testowych}

\begin{figure}[H]
    \centering
    \includegraphics[width=0.5\linewidth]{images/plot1.png}
    \caption{Wielokąt bazowy}
    \label{fig:wlasny_tri}
    \vspace{0.5em}
    \textit{Komentarz: Prosty, niesymetryczny wielokąt $y$-monotoniczny. Algorytm poprawnie połączył wierzchołki.}
\end{figure}

\begin{figure}[H]
    \centering
    \includegraphics[width=0.5\linewidth]{images/plot10.png}
    \caption{Wachlarz}
    \label{fig:wachlarz_tri}
    \vspace{0.5em}
    \textit{Komentarz: Jest to najprostszy przypadek wklęsłości (jeden łańcuch jest prostą linią), w którym algorytm tworzy wachlarz trójkątów z jednym wspólnym wierzchołkiem. Triangulacja została poprawnie wykonana.}
\end{figure}

\begin{figure}[H]
    \centering
    \includegraphics[width=0.5\linewidth]{images/plot3.png}
    \caption{Choinka}
    \label{fig:choinka_tri}
    \vspace{0.5em}
    \textit{Komentarz: Wielokąt z podobnymi wklęsłościami po obu stronach. Algorytm pomyślnie poradził sobie z zarządzaniem oboma łańcuchami, tworząc trójkąty w miarę wychodzenia w górę (wzrostu $y$).}
\end{figure}



\begin{figure}[H]
    \centering
    \includegraphics[width=0.5\linewidth]{images/plot7.png}
    \caption{Grzebień}
    \label{fig:grzebien_tri}
    \vspace{0.5em}
    \textit{Komentarz: Reprezentuje trudniejszy przypadek, w którym występuje wiele następujących po sobie wklęsłych narożników na jednym z łańcuchów. Również tutaj, algorytm sobie poradził}
\end{figure}


\section{Wnioski}
Implementacja algorytmu triangulacji wielokątów $y$-monotonicznych przebiegła pomyślnie. 
Algorytm triangulacji, bazujący na
sortowaniu wierzchołków i wykorzystaniu stosu, efektywnie i optymalnie czasowo ($\mathbf{O(n)}$) generuje
zbiór przekątnych, poprawnie dzieląc dostarczone wielokąty $y$-monotoniczne na trójkąty.

\section{Środowisko}
\begin{itemize}
    \item System operacyjny: Linux 6.17.9-arch1-1
    \item Procesor: AMD Ryzen 9800x3d
    \item Język programowania + translator: Python 3.13.7
    \item Środowisko: JupyterNotebook
    \item wynik !jupyter --version
        \begin{itemize}
        \item IPython          : 9.7.0
        \item ipykernel        : 7.1.0
        \item ipywidgets       : not installed
        \item jupyter\_client   : 8.6.3
        \item jupyter\_core     : 5.9.1
        \item jupyter\_server   : 2.17.0
        \item jupyterlab       : 4.5.0
        \item nbclient         : 0.10.2
        \item nbconvert        : 7.16.6
        \item nbformat         : 5.10.4
        \item notebook         : 7.5.0
        \item qtconsole        : not installed
        \item traitlets        : 5.14.3
    \end{itemize}
\end{itemize}
\end{document}
